\documentclass{article}
\title{Quantum Now}
\usepackage{multicol}	%Content in multiple columns

\begin{document}
	\pagenumbering{roman}
	\setlength{\columnsep}{20pt}
	\setlength{\columnseprule}{1pt}

	\maketitle
	\tableofcontents

	\section{Introduction}
	\paragraph{}
	Something, something.

	\section{Python Functions}
	\paragraph{}
	More something, more something
		\subsection{Modulus Function}
		\paragraph{}
		At Python function
			\subsubsection{Mathematics}
			Some some
			\subsubsection{Pseudocode}
			Some
			\subsubsection{Code}
			Some some some
		\subsection{Absolute Value Function}
		\paragraph{}
		This function is made to accept any real number and return the positive value if it is not already.
			\subsubsection{Mathematics}
			Let A be a real number, the absolute value of A is A been a positive number.
			\subsubsection{Pseudocode}
			If A < 0: multiply A by -1 and return the result.
			If A >= 0: return A as it is.
			\subsubsection{Python Code}
			\begin{verbatim}
				DO i=1,5
					print*, 'Do something'
				END DO
			\end{verbatim}
			\subsubsection{Fortran Code}

		\subsection{Modulus Function}
		\paragraph{}
		At Python function

		\subsection{Division Function} \label{sec:df}
		\paragraph{}
		At Python function

		\subsection{Sine/Cosine Function}
		\paragraph{}
		At Python function

		\subsection{Square root Function}
		\paragraph{}
		In Sec.~\ref{sec:df} the division algorithm was fully described but we encourage you to review that section as we would be using that algorithm in this function.

	\section{Array Functions}
	\paragraph{}
	More more something, more more something

	\section{Applications}
	\paragraph{}
	More more more something, more more more something

	%
\end{document}

%\tableofcontents
